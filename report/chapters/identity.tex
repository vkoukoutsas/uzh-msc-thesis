\chapter{Blockchain Identity Management}

\section{IAM Requirements} \label{ch:educhain-iam-reqs}

\section{Roles} \label{ch:educhain-roles}

To satisfy identity related REQs as mentioned in Educhain-Common \cite{educhain-common}. The following roles have been identified:

\begin{description}
	\item[Issuer] \hfill \\
	The Issuer is an individual that is officially associated with an issuing organization (as defined in Educhain-commons or another chapter in this report \textbf{TBD}) and has been granted the right to issue digital certificates on behalf of this organization.
	\item[Recipient]
	\item[Verifier]
	\item[System Administrator]  
\end{description}


Roles description:
- Issuer Anyone that can Issue a digital certificate (diploma, work certificate etc.) on behalf of an issuing Organization (school, university, Company etc.).

\section{IAM Design}

\subsection{Candidate Design Solutions}

\subsubsection{Dedicated IdP}

\subsubsection{External IdP}

Describe Edu-ID. (Should edu-id be mentioned first here or in the Related Work?)

What is it, what benefits it provides.

Explain what is an SP.

Explain Attribute-based Access Control (ABAC) and SSO (SAML).


\section{Educhain IAM Solution}

1. Onboarding of Educhain as an SP
2. Technical Integration
3. User Registration to Edu-ID
4. User Login
5. High level description of Access control (identification, Authentication, Authorization)

Here use examples from SWITCH and put design on CH4 and Implementation on CH5