\chapter*{Abstract}
\addcontentsline{toc}{chapter}{Abstract}

\selectlanguage{english}

Academic certificates have major relevance in the labor market, signaling capability, and the level of education and skills of the recipient. Unfortunately, recent years have seen an increase in fraud, ranging from inflating academic grades to fake diplomas. A countermeasure use-case applicable to academic certificates is Proof-of-Existence (PoE), which effectively timestamps a certificate, thus, proving the existence of exactly this certificate, without leaking information about its content (the certificate's data). Prior work presented the initial requirements for a solution targeted at the University of Zurich \cite{educhain-architecture}, an essential building block of such a solution is identity and access management (IAM). \\
The goal of this master's thesis is to design and implement a suitable private distributed ledger (DL) solution with an integrated identity and access management module. The resulting solution is intended to be used in the Swiss Educhain service \cite{educhain-proposal} to satisfy the main requirement of the various stakeholders, which is the issuance and verification of digital certificates utilizing blockchain technology. The Proof-of-Concept (PoC) implementation is evaluated against the identified requirements and the prototype's functionality. This master's thesis provides a blockchain-based identity and access management solution as an integral part of the produced Swiss Educhain PoC implementation.


%e.g., by first generating a unique cryptographic hash digest of a certificate and then publishing that hash to a public blockchain,

\chapter*{Zusammenfassung}
\selectlanguage{ngerman}

Akademische Abschl{\"u}sse haben eine grosse Bedeutung f{\"u}r den Arbeitsmarkt, da sie die F{\"a}higkeit und das Bildungsniveau des Empf{\"a}ngers signalisieren. Leider haben in den letzten Jahren die Betrugsf{\"a}lle mit gefaelschten Diplomen zugenommen. Eine Gegenmassnahme, die auf akademische Zertifikate anwendbar ist, ist Proof-of-Existence (PoE). Mit PoE wird ein Zertifikat mit einem Zeitstempel versehen und so die Existenz genau dieses Zertifikats nachgewiesen, ohne dass Informationen {\"u}ber den Inhalt durchsickern. In vorhergegangenen Arbeiten wurden die ersten Anforderungen f{\"u}r eine L{\"o}sung an der Universit{\"a}t Z{\"u}rich vorgestellt \cite{educhain-architecture}. Ein wesentlicher Bestandteil einer solchen L{\"o}sung ist das Identit{\"a}ts- und Zugriffsmanagement. \\
Das Ziel dieser Masterarbeit ist es, eine geeignete L{\"o}sung mit einem Distributed Ledger (DL) und integriertem Identit{\"a}ts- und Zugriffsmanagementmodul zu entwerfen und zu implementieren. Die daraus resultierende L{\"o}sung soll im Swiss Educhain Projekt \cite{educhain-proposal} eingesetzt werden, um die Hauptanforderung der verschiedenen Requirements zu erf{\"u}llen, dazu geh{\"o}rt die Ausstellung und Verifizierung von digitalen Zertifikaten unter Verwendung der Blockchaintechnologie. Die Implementierung eines Proof-of-Concepts (PoC) wird anhand der identifizierten Requirements und der Funktionalit{\"a}t des Prototyps evaluiert. Diese Masterarbeit bietet eine blockchainbasierte Identit{\"a}ts- und Zugriffsmanagementl{\"o}sung als integralen Bestandteil der resultierenden Swiss Educhain PoC Implementierung.



\selectlanguage{english}